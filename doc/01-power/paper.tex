\documentclass[11pt, letterpaper]{article}
	\title{Energy-Efficient Real-Time Systems}
	\author{Daniel Lo \{dl575\}}
	
	% Packages for math formatting
	\usepackage{amsfonts}
	\usepackage{amsmath}
	\usepackage{amsthm}

	% for letter paper
	\usepackage[letterpaper]{geometry}
	% 1 inch margins
	\usepackage{fullpage}    
	% allow enumerate numberings to be specified
	\usepackage{enumerate}
	% Insert images
	\usepackage{graphicx}
	% For syntax highlighting
	\usepackage{listings}
  % Graphviz
  \usepackage{graphviz}
  % Algorithms
  \usepackage{algorithm}
  \usepackage{algorithmic}
		
	% Header
	\usepackage{fancyhdr}
	\pagestyle{fancy}
	\headheight 30pt
	\rhead{}
	\lhead{Energy-Efficient Real-Time Systems \\
  Daniel Lo \{dl575@cornell.edu\}}
  \headsep 0.1in
	
	% Define a problem "theorem" heading
	\newtheorem{problem}{Problem}
	% Proposition theorem
	\newtheorem{proposition}{Proposition}
	% Scientific form
	\providecommand{\e}[1]{\ensuremath{\cdot 10^{#1}}}
	% Matrix - use mathbf font
	\providecommand{\m}[1]{\mathbf{#1}}
	% Degree symbol
	\providecommand{\degrees}{^{\circ}}
	% Insert figure
	\providecommand{\fig}[1]{
		\noindent
		\begin{center}
			\includegraphics[height=3.5in]{#1}
		\end{center}
	}
  \providecommand{\figcol}[1]{
    \noindent
    \begin{center}
      \includegraphics[width=\columnwidth]{#1}
    \end{center}
  }
	\providecommand{\dualfig}[2]{
		\noindent
		\begin{center}
		\includegraphics[width=3.2in]{#1} \hspace{-0.3in}
		\includegraphics[width=3.2in]{#2}
		\end{center}
	}
	\providecommand{\tripfig}[3]{
		\noindent
		\begin{center}
		\hspace{-0.3in}
		\includegraphics[width=2.3in]{#1} \hspace{-0.3in}
		\includegraphics[width=2.3in]{#2} \hspace{-0.3in}
		\includegraphics[width=2.3in]{#3} \hspace{-0.3in}
		\end{center}
	}
	% begin/end align*
	\providecommand{\eq}[1]{
		\begin{align*}
		#1
		\end{align*}
	}
	% Kernel/Image
	\providecommand{\im}[1]{
		\text{Im}(#1)
	}
	\renewcommand{\ker}[1]{
		\text{Ker}(#1)
	}
	% Trace
	\providecommand{\tr}[1]{
		\text{Tr}\left(#1\right)
	}
	% Such that
	\providecommand{\st}[0]{
		%\text{ s.t. }
		\ni
	}
	% Real numbers
	\providecommand{\reals}[0]{
		\mathbb{R}
	}
	% Use overline which is longer instead of bar
	\renewcommand{\bar}[1]{
		\overline{#1}
	}
	% Create divided matrix representing state-space system
	\providecommand{\statespace}[4]{
		\begin{bmatrix}
		\begin{array}{c|c}
			#1 & #2 \\
			\hline
			#3 & #4
		\end{array}
		\end{bmatrix}
	}
	% Include matlab code
	\providecommand{\matlab}[1]{
	  \lstinputlisting[language=matlab,
		showstringspaces=false,
		basicstyle=\footnotesize]
		{#1}
	}
  % Generate graph using dot/graphviz
  \providecommand{\dotgraph}[2]{
    \vspace{-0.6in}
    \begin{center}
    \digraph[scale=0.75]{#1}{#2}
    \end{center}
    \vspace{-0.6in}
  }
		
	
	% Insert a blank line between paragraphs
	\setlength{\parskip}{\baselineskip}
\begin{document}

%\maketitle

\section{Introduction}

This document discusses plans and ideas dealing with the idea of using
execution time prediction to inform the use of DVFS in soft real-time (e.g.,
interactive) systems.

\section{Power Modeling}

We need some way to model what the power/energy usage for executing a job at a
specified DVFS point is. Consider the following simple model of core power.

\eq{
  P_{static} =& I_{leak}V \\
  P_{dynamic} =& \alpha C V^2 f \\
  P =& P_{static} + P_{dynamic} \\
  =& I_{leak}V + \alpha C V^2 f 
}

Let's begin by only considering the dynamic power. The dynamic component of the energy used by a job that takes time $t$ is then
\eq{
  E_{dynamic} = \alpha C V^2 f t
}
If we assume that the job time is exactly proportional to frequency ($t = \beta/f$) then,
\eq{
  E_{dynamic} = \alpha \beta C V^2
}

\subsection{Voltage versus Frequency Relationship}
\label{sec:voltage_vs_frequency}

A common assumption is that voltage and frequency can be scaled proportionally
(e.g., [SleepScale, ISCA 2014]). In this section, we attempt to see how
reasonable this is based on information from processor datasheets.

For the OMAP 3530 which runs a Cortex-A8
\footnote{http://www.ti.com/lit/ds/symlink/omap3530.pdf}, the following table
lists the recommended operating points.

\begin{tabular}{|l|r|r|}
\hline
Operating Point & Frequency [MHz] & Voltage [V] \\ \hline\hline
OPP6 & 720 & 1.35  \\ \hline   
%OPP5 & 600 & 1.35  \\ \hline   
OPP4 & 550 & 1.27  \\ \hline   
OPP3 & 500 & 1.20  \\ \hline   
OPP2 & 250 & 1.06  \\ \hline   
OPP1 & 125 & 0.985 \\ \hline   
\end{tabular}

Lower-speed versions of the OMAP 3530 have an operating point OPP5 instead of OPP6 which runs at 600 MHz at 1.35 V. We assume the high-speed version that runs at 720 MHz at 1.35 V here.

The operating points for the Pentium M
\footnote{http://download.intel.com/design/network/papers/30117401.pdf} are
shown in the following table.

\begin{tabular}{|l|r|r|}
\hline
Operating Point & Frequency [MHz] & Voltage [V] \\ \hline\hline
P0 & 1600 & 1.484 \\ \hline   
P1 & 1400 & 1.420 \\ \hline   
P2 & 1200 & 1.276 \\ \hline   
P3 & 1000 & 1.164 \\ \hline   
P4 & 800  & 1.036 \\ \hline   
P5 & 600  & 0.956 \\ \hline   
\end{tabular}

The power versus frequency operating points for the Intel Atom Z2480
\footnote{http://pascal.computer.org/csdl/mags/mi/2013/06/mmi2013060038.pdf} is
shown in the following table. We calculate a ``scaled voltage'' based on the power number
using $V = \sqrt{P/f}$ where the power is in Watts and the frequency is in Hz.

\begin{tabular}{|l|r|r|r|}
\hline
Operating Point & Frequency [MHz] & Power [mW] & Scaled Voltage \\ \hline\hline
P0 & 2000 & 1200 & 0.024\\ \hline   
P1 & 1300 & 500 & 0.020 \\ \hline   
P2 & 600 & 175 & 0.017 \\ \hline   
P3 & 100 & 50 & 2.2\e{-5} \\ \hline   
\end{tabular}

The following graphs show the relationship between frequency and voltage. From
left to right these show the data for the OMAP, the Pentium M, and the Atom. We
use Excel to add a linear trendline for each processor.
\tripfig{figs/omap.pdf}{figs/pentiumm.pdf}{figs/medfield.pdf}
We see a strong linear relationship between voltage and frequency for the OMAP
and the Pentium M ($R = 0.99$). The data for the Atom still shows a linear
relationship, though not as strong ($R = 0.79$). However, note that the Atom data was
derived from power numbers which includes both static and dynamic power.

For the OMAP and the Pentium M, we see a strong linear relationship. However,
note that the relationship is not proportional. There exists an offset in the
trendline that Excel derives (i.e., $V = af + b$). Using this in our equation for dynamic power, we have
\eq{
  P &= \alpha CV^2f \\
  &= \alpha C(af + b)^2f \\
  &= \alpha C(a^2f^3 + 2abf^2 + f)
}

To first order, we can consider only the dominant term $f^3$, in which case the
result is equivalent to the assumption that voltage and frequency scale
proportionally.

\subsubsection{Other Data Points}

Zhu and Reddi \footnote{Yuhao Zhu and Vijay Janapa Reddi. High-Performance and
Energy-Efficient Mobile Web Browsing on Big/Little Systems. HPCA 2013.} have
found frequency and voltage information for the OMAP 4460 (Cortex-A9) and OMAP3
(Cortex-A8).

For the Cortex-A9

\begin{tabular}{|l|r|r|}
\hline
Frequency [MHz] & Voltage [V] \\ \hline\hline
300 & 0.83 \\ \hline
700 & 1.01 \\ \hline
920 & 1.11 \\ \hline
1200 & 1.27 \\ \hline
\end{tabular}

For the Cortex-A8

\begin{tabular}{|l|r|r|}
\hline
Frequency [MHz] & Voltage [V] \\ \hline\hline
300 & 0.94 \\ \hline
600 & 1.10 \\ \hline
800 & 1.26 \\ \hline
\end{tabular}

\subsection{Race-to-Idle}

Race-to-idle has recently been proposed as a way to save energy. The idea is to
run at the maximum frequency and finish jobs quickly. This allows the core to
shut down eliminating energy usage due to static power. If static power is
large enough compared to dynamic power than this is advantageous.
From our power model,

\eq{
  E_{static} &= I_{leak}Vt \\
  E_{dynamic} &= \alpha C V^2ft \\
  &= \alpha\beta C V^2 \\
  E &= IVt + \alpha \beta CV^2
}

Running faster requires increasing voltage which increases the energy
contribution from the second (dynamic) term. However, it decreases the job time
$t$ in the first term. On the other hand, running slower increases $t$ but
decreases $V$. Depending on the values of $I$, $\alpha$, $\beta$, $C$, and the
relationship between voltage and execution time, race-to-idle may or may not be
advantageous.


% \subsubsection{DVFS Data}
% 
% For the OMAP 3530 which runs a Cortex-A8
% \footnote{http://www.ti.com/lit/ds/symlink/omap3530.pdf}, the following table
% lists the recommended operating points.
% 
% \begin{tabular}{|l|r|r|}
% \hline
% Operating Point & Voltage [V] & Frequency [MHz] \\ \hline\hline
% OPP6 & 1.35 & 720 \\ \hline   
% OPP5 & 1.35 & 600 \\ \hline   
% OPP4 & 1.27 & 550 \\ \hline   
% OPP3 & 1.20 & 500 \\ \hline   
% OPP2 & 1.06 & 250 \\ \hline   
% OPP1 & 0.985 & 125 \\ \hline   
% \end{tabular}
% 
% OPP6 is supported on high-speed versions of the OMAP 3530 and we will assume
% this version for our analysis. The following is a graph of these DVFS
% operating points.
% 
% \fig{dvfs.pdf}
% 
% We normalize energy usage and execution times to the values at OPP6. These
% are shown in the following table and graph. 
% 
% % OMAP 3530
% \begin{tabular}{|l|r|r|}
% \hline
% Operating Point & Normalized Energy & Normalized Execution Time \\ \hline\hline
% OPP6 & 1.00 & 1.00 \\ \hline
% OPP4 & 0.88 & 1.31 \\ \hline
% OPP3 & 0.79 & 1.44 \\ \hline
% OPP2 & 0.62 & 2.88 \\ \hline
% OPP1 & 0.53 & 5.76 \\ \hline
% \end{tabular}
% 
% \fig{dvfs_energy.pdf}

\section{Execution Time Data}

The following is the execution time for performing video conversion on each
frame of ironman3.mp4 using ffmpeg. This was run on a 2013 Macbook Air running a 1.7 GHz Intel Core i7. The minimum time is 2.9 ms. The average
time is 5.1 ms. The maximum time is 33.4 ms. There are 3659 frames.
\figcol{frame_time.pdf}

We will assume that this data corresponds to operating at the highest operating
point (i.e., P0) and the execution time requirement is 35 ms (just above the
worst-case seen).

\subsection{DVFS Schemes}

\textbf{Naive Schemes: }
A naive scheme to meet the execution time would be to set the DVFS point at the
fastest operating point in order to account for the worst-case frame time. Let
$E_i = 1$ be the energy for processing frame $i$ at P0, then the total energy
usage for this worst-case scheme is $E = 3659$ and no frames violate the timing
requirement. We could use similarly naive schemes that use a constant DVFS
point. The energy usage and percentage of frames which violate timing are
listed in the following table if we assume the operating points for the Pentium M.

\begin{tabular}{|l|r|r|}
\hline
Operating Point & Energy & Frames Violated \\ \hline\hline
P0 & 3659 (100\%) & 0 (0\%) \\ \hline
P1 & 3350 (91\%) & 0 (0.00\%) \\ \hline
P2 & 2705 (73\%) & 1 (0.03\%) \\ \hline
P3 & 2251 (61\%) & 2 (0.05\%) \\ \hline
P4 & 1783 (48\%) & 2 (0.05\%) \\ \hline
P5 & 1518 (41\%) & 2 (0.05\%) \\ \hline
P-average & 935 (25\%) & 1083 (29.6\%) \\ \hline
\end{tabular}

For the DVFS points we consider here, even at the lowest setting, most frames meet the required response-time.
This is because there is a 33.4 ms/5.1 ms = 6.5x gap between the worst-case and
the average-case execution time. However, for frequencies there is only a
1600/600 = 2.7x difference. Suppose we actually had a frequency corresponding
to the average-case execution time. A frequency of 250 MHz corresponds to a
6.4x ratio from the max frequency. Using the trendline from
Section~\ref{sec:voltage_vs_frequency}, the corresponding voltage is
$0.0006(250) + 0.6 = 0.75 V$. If this average-case situation is used, then the
energy is 935 (25\%) but 1083 (29.6\%) frames violate the response-time
requirement.

\textbf{Optimal (Oracle) Scheme: } 
Suppose we have a continuous range of DVFS settings and that DVFS settings can
reach arbitrarily low voltages and frequencies. If we select the minimal
DVFS point for each frame such that the response-time requirement is met, then
the resulting total energy we find is 932 (25\%). This is close to the energy
usage found for using a DVFS setting accounting for the average-case behavior
while still meeting all deadlines.

% If we choose the lowest voltage setting for each frame such that the 40ms
% timing requirement is still met, then the total energy is 1952 which is 53\% of
% the energy usage of running all frames at OPP6. This gives the energy of
% running at OPP1 but with no frames violated as in running at OPP6.
% 
% Note that with the DVFS settings we explore here, for the average frame which
% ran in 4.9ms at OPP6, its execution time at OPP1 is 28.2ms which is still
% within our timing requirement. With lower DVFS points (or tighter timing
% requirements), greater energy savings are possible.

\textbf{Adaptive Scheme: }
A simple adaptive scheme is to adapt the DVFS setting based on previous frame times.
Again, we assume a continuous and infinite range of DVFS settings. Since there
does exist some noise between frames, we will use 1.2x the execution time of
the previous frame to set the DVFS setting for the current frame. With this, we
see a total energy usage of 1003 (27\%) which is close to the average-case and
optimal-case results. However, the number of frames violated is 481 (13\%).
Thus, we see that using the previous frame to predict the operating point
performs poorly.

An improvement on this adaptive scheme would be use some window of past frame
times to inform our DVFS setting. We use a window of the past 10 frames and
then take 1.2x the max time in this window to choose our DVFS setting for the
current frame. The result is a total energy usage of 1106 (30\%) with 32
(0.87\%) frames violated. We are still not able to meet all deadlines and the
energy usage is 19\% more than optimal.

\end{document}
